\documentclass{article}

% formato de página
\usepackage[margin = 1.5cm, letterpaper]{geometry}
\usepackage[spanish,es-noshorthands]{babel}

% formta de ecuaciones
\usepackage{amsmath}

\begin{document}
\title{
    Autómatas y Lenguajes formales \\
    Ejercicio Semanal 12
}

\author{
    Sandra del Mar Soto Corderi \\
    Edgar Quiroz Castañeda
}

\date{
    14 de mayo del 2019
}

\maketitle

\begin{enumerate}
    \item Transforma la siguiente gramática a \textbf{FNC}.
    \begin{align*}
        &S \rightarrow \ 0S1 |\ A |\ AB \\
        &A \rightarrow \ 1A0 |\ S\varepsilon \\
        &B \rightarrow \ 0B |\ 1C \\
        &C \rightarrow \ 0C |\ 0 |\ vacio \\
        &D \rightarrow \ 0C |\ 1D |\ F \\
        &F \rightarrow \ 0F |\ \varepsilon
    \end{align*}
    \begin{enumerate}
        \item Primero, hay que eliminar las variables inútiles.
        
        Primero, las alcanzables.
        \begin{enumerate}
            \item $Acc ::= \{S\}$
            \item $Acc :: = \{S\} \cup \{S, A, B\}$
            \item $Acc :: = \{S, A, B\} \cup \{S, A, B, C\}$
            \item $Acc :: = \{S, A, B, C\} \cup \{S, A, B, C\}$
        \end{enumerate}

        Por lo que las no alcanzables son $\{S, A, B, C\}^{c} = \{D, F\}$

        Luego, las productivas
        \begin{enumerate}
            \item $Prod ::= \{C\}$
            \item $Prod ::= \{C\} \cup \{C, D, B\}$
            \item $Prod ::= \{C, D, B\} \cup \{C, D, B, S\}$
            \item $Prod ::= \{C, D, B, S\} \cup \{C, D, B, S, A\}$
            \item $Prod ::= \{C, D, B, S, A\} \cup \{C, D, B, S, A\}$
        \end{enumerate}

        Por lo que las no productivas son $\{C, D, B, S, A\}^{c} = \{F\}$

        Entonces, para las variables no inútiles son
        $\{S, A, B, C, D, F\} \setminus \{F\} \setminus \{D, F\}
        = \{S, A, B, C\}$

        Entonces la gramática en este punto es 

        \begin{align*}
            &S \rightarrow \ 0S1 |\ A |\ AB \\
            &A \rightarrow \ 1A0 |\ S \\
            &B \rightarrow \ 0B |\ 1C \\
            &C \rightarrow \ 0C |\ 0 |\ vacio \\
        \end{align*}

        \item Hay que eliminar las transiciones que lleven al símbolo inicial.
        
        Hay que aladir $S_0 \rightarrow S$, y el nuevo símbolo inicial sea $S_0$.

        La gramática en este punto es

        \begin{align*}
            &S_0 \rightarrow \ S \\
            &S \rightarrow \ 0S1 |\ A |\ AB \\
            &A \rightarrow \ 1A0 |\ S \\
            &B \rightarrow \ 0B |\ 1C \\
            &C \rightarrow \ 0C |\ 0 |\ vacio \\
        \end{align*}

        \item Sustituir las reglas que producen cadenas con símbolo terminales y
        no terminales

        $S \rightarrow \ 0S1$ se convierte en $S \rightarrow \ N_0SN_1$.

        $A \rightarrow \ 1A0 |\ S$ se convierte en 
        $A \rightarrow \ N_1AN_0$.

        $B \rightarrow \ 0B |\ 1C$ se convierte en $B \rightarrow \ N_0B |\ N_1C$.

        $C \rightarrow \ 0C$ se convierte en $C \rightarrow \ N_0C$.
        
        Y hay que añadir las reglas $N_0 \rightarrow \ 0 $ y $N_1 \rightarrow \ 1$.

        La gramática en este punto es
        \begin{align*}
            &S_0 \rightarrow \ S \\
            &S \rightarrow \ N_0SN_1 |\ A |\ AB \\
            &A \rightarrow \ N_1AN_0 |\ S \\
            &B \rightarrow \ N_0B |\ N_1C \\
            &C \rightarrow \ N_0C |\ N_0 |\ vacio \\
            &N_0 \rightarrow \ 0 \\
            &N_1 \rightarrow \ 1
        \end{align*}

        \item Eliminar las reglas con más de dos símbolo no terminales a la derecha.
        
        $S \rightarrow \ N_0SN_1$ se convierte en $S \rightarrow \ N_0S_1$ y 
        $S_1 \rightarrow \ SN_1$.

        $A \rightarrow \ N_1AN_0$ se convierte en $A \rightarrow \ N_1A_1$ y 
        $A_1 \rightarrow \ AN_0$.

        La gramática en este punto es
        \begin{align*}
            &S_0 \rightarrow \ S \\
            &S \rightarrow \ N_0S_1 |\ A |\ AB \\
            &S_1 \rightarrow \ SN_1 \\
            &A \rightarrow \ N_1A_1 |\ S \\
            &A_1 \rightarrow \ AN_0 \\
            &B \rightarrow \ N_0B |\ N_1C \\
            &C \rightarrow \ N_0C |\ N_0 |\ vacio \\
            &N_0 \rightarrow \ 0 \\
            &N_1 \rightarrow \ 1
        \end{align*}

        \item Eliminar las reglas que produzcan $\varepsilon$ cuando el símbolo
        a la izquierda no sea inicial.

        No hay.

        \item Eliminar las reglas de producción unitarias.
        
        Primero, hay que encontrar las variables unitarias de cada vaiarble.
        \begin{enumerate}
            \item $Unit(S) ::= \{S\}$
            \item $Unit(S) ::= \{S\} \cup \{A\}$
            \item $Unit(S) ::= \{S, A\} \cup \{S, A\}$
        \end{enumerate}

        Entonces, hay que remplazar $S \rightarrow A$ por $S \rightarrow N_1A_1$.

        \begin{enumerate}
            \item $Unit(A) ::= \{A\}$
            \item $Unit(A) ::= \{A\} \cup \{S\}$
            \item $Unit(A) ::= \{A, S\} \cup \{A, S\}$
        \end{enumerate}

        Entonces, hay que remplazar $A \rightarrow S$ por $A \rightarrow N_0S_1| \ AB$.

        \begin{enumerate}
            \item $Unit(B) ::= \{B\}$
            \item $Unit(B) ::= \{B\} \cup \emptyset$
        \end{enumerate}
        No hay que cambiar nada.

        \begin{enumerate}
            \item $Unit(C) ::= \{C\}$
            \item $Unit(C) ::= \{C\} \cup \emptyset$
        \end{enumerate}
        No hay que cambiar nada.

        \begin{enumerate}
            \item $Unit(N_0) ::= \{N_0\}$
            \item $Unit(N_0) ::= \{N_0\} \cup \{C\}$
            \item $Unit(N_0) ::= \{N_0, C\} \cup \{N_0, C\}$
        \end{enumerate}
        Entonces, hay que remplazar $C \rightarrow N_0$ por $C \rightarrow 0$.

        \begin{enumerate}
            \item $Unit(N_1) ::= \{N_1\}$
            \item $Unit(N_1) ::= \{N_1\} \cup \emptyset$
        \end{enumerate}
        No hay que cambiar nada.

        Entonces la gramática en este punto es 
        \begin{align*}
            &S_0 \rightarrow \ S \\
            &S \rightarrow \ N_0S_1 |\ N_1A_1 |\ AB \\
            &S_1 \rightarrow \ SN_1 \\
            &A \rightarrow \ N_1A_1 |\ N_0S_1| \ AB \\
            &A_1 \rightarrow \ AN_0 \\
            &B \rightarrow \ N_0B |\ N_1C \\
            &C \rightarrow \ N_0C |\ 0 |\ vacio \\
            &N_0 \rightarrow \ 0 \\
            &N_1 \rightarrow \ 1
        \end{align*}
    \end{enumerate}
\end{enumerate}
\end{document}